\documentclass[a4paper, 12pt]{article}
\textwidth 16.5 cm
\oddsidemargin 0 cm
\topmargin -1.5 cm
\textheight 25 cm
\renewcommand{\baselinestretch}{1.25}
\markright{Alfredo Ibias Martínez y Daniel Loscos Barroso}
\usepackage{amsmath,amssymb,amsfonts,latexsym,cancel}
\usepackage[spanish]{babel}
\usepackage[utf8]{inputenc}
\usepackage[T1]{fontenc}
\usepackage{graphicx}
\usepackage{makeidx}
\usepackage{float}
\usepackage{color}
\makeindex
\newcommand{\sen}{\mathop{\rm sen}\nolimits}
\newcommand{\arcsen}{\mathop{\rm arcsen}\nolimits}
\newcommand{\arcsec}{\mathop{\rm arcsec}\nolimits}
\newcommand{\R}{\mathbb{R}}
\newcommand{\N}{\mathbb{N}}
\newcommand{\Z}{\mathbb{Z}}
\def\max{\mathop{\mbox{\rm m\’ax}}}
\def\min{\mathop{\mbox{\rm m\’in}}}

\begin{document}
\title{\Huge \bf Especificación del lenguaje \textit{Space}}
\author{
	\small Alfredo Ibias Martínez\\
	\small Daniel Loscos Barroso\\
    \small Doble Grado Matemáticas - Ingeniería Informática
    }
\date{}
\maketitle

\vfill
\begin{center}
{\footnotesize Trabajo realizado con \LaTeX}
\end{center}
\newpage

\begin{center}
\tableofcontents
\end{center}
\newpage

\section{Presentación del lenguaje}

Para guiarnos en la especificación del lenguaje y en la elección de los requisitos que queremos que cumpla vamos a emplear la siguiente motivación: Queremos crear un lenguaje pensado para su uso en misiones espaciales. Su utilidad principal será crear los sistemas de control y soporte de los satélites y naves lanzadas al espacio. Por este motivo hemos decidido llamarlo \textit{Space}.\\

Deberá ser un lenguaje sencillo y ligero, que evite lo más posible la generación de errores, y que esté fuertemente orientado a la ejecución de órdenes de forma secuencial. Es especialmente importante la minimización de los errores pues estos, por pequeños que sean, pueden llegar a causar pérdidas millonarias.\\

Por lo tanto, este lenguaje será un lenguaje imperativo básico, con una sintaxis simple y pocas diferencias con \textit{C}, para que se pueda empezar a usar rápidamente si ya se conoce este lenguaje de programación.\\

\section{Requisitos del lenguaje}
El lenguaje cumplirá los siguientes requisitos:
\subsection{Identificadores y ámbitos de definición}
El lenguaje permitirá la declaración de variables simples y de arrays de cualquier tipo, incluidos arrays de arrays. Permitirá también bloques anidados (con su tabla de símbolos correspondiente) y permitirá crear funciones y procedimientos para simplificar y mejorar la legibilidad de los programas.\\

No incluirá punteros ni registros para no permitir el acceso a la memoria dinámica ni las referencias a memoria. El objetivo es evitar errores que pongan en peligro la integridad de partes sensibles de la memoria. Tampoco permitirá clases, módulos ni cláusulas de importación para evitar errores derivados de la programación modular y para simplificar los programas. Si se quiere usar alguna función de biblioteca, esta deberá ser copiada y pegada directamente en el código del programa.
\subsection{Tipos}
Se tendrá declaración explícita del tipo de las variables y los tipos predefinidos (integer, real, boolean, character, Numeric y Type).\\

El lenguaje contendrá operadores infijos, con distintas prioridades y asociatividades, para realizar operaciones con los distintos tipos. Se podrá tener también tipos sin nombre y se podrán definir tipos de usuario de forma limitada (en concreto, sólo enumerados).\\

La equivalencia de tipos será por nombres, para evitar problemas.

\subsection{Instrucciones ejecutables}
\textit{Space} Tendrá instrucciones de: 
\begin{itemize}
\item Asignación, incluyendo arrays. 
\item Condicional con una y dos ramas.
\item Bucle. Se tendrá el típico bucle while.
\item Instrucción case con salto a cada rama en tiempo constante.
\item Expresiones formadas por constantes, identificadores (con y sin subíndice) y operadores infijos.
\item Llamadas a procedimientos y funciones.
\end{itemize}

Lo que no se tendrá serán expresiones con punteros y nombres cualificados, ni instrucciones de reserva o liberación de memoria dinámica para evitar los problemas con punteros y memoria dinámica señalados anteriormente.
\subsection{Errores}
El compilador indicará el tipo de error, y las fila y columna donde se ha producido. Acto seguido parará la compilación.
\newpage

\section{Sintaxis}
La sintaxis del lenguaje se define a continuación:
\subsection{Unidades Léxicas}
Estas son las unidades léxicas que hemos contemplado:
\begin{align*}
%Vale, todo esto sale en cursiva, podemos ahorrarnos miles de  \textit
&letraMin = \textrm{a|...|z}\\
&letraMay = \textrm{A|...|Z}\\
&letra =letraMay|letraMin\\
&digitoPos = \textrm{1|...|9}\\
&digito = digitoPos|\textrm{0}\\
&parteEnt = digitoPos(digito)^*\\
&parteDec = (digito)^*digitoPos\\
&int =parteEnt|\textrm{0}\\
&real =int|(int).parteDec\\
&separador=\textrm{\textbackslash t|\textbackslash  n}\\
&simbolo= +|-|=|>|\%|...|\textbackslash|.|,|;\\
&comentario=\textbackslash \textbackslash (letra|digito|\textbackslash \textrm{t}|simbolo)^* \textbackslash \textrm{n}\\
&identificadorVar =letraMin(digito|letra|\_)^*([parteEnt|\textrm{0}])^*\\
&char =\ \textrm{'}(letra|digito|simbolo|separador)\textrm{'}\\
&num =int|real\\
&bool =true|false\\
&type =num|char|bool\\
&argumento =type|identificadorVar\\
&identificadorFun =letraMay(digito|letra|\_)^*((\varepsilon|(argumento(,argumento)^*))\\
&identificador =identificadorVar|identificadorFun\\
&palabrasReservadas= \textrm{while|if|else|...|float|bool|int|...|return|DEFINE|DECLARE|START|END}\\
\end{align*}

\subsection{Generalidades}
Todo programa empieza en START y ejecuta las instrucciones que encuentra hasta llegar a END. Sin embargo el código no empieza ahí, tendremos tres secciones de código:
\begin{itemize}
\item DECLARE: Es el primer bloque de código, en él se declaran las variables globales, las funciones y los procedimientos.
\item DEFINE: Es la sección del código donde se definen las funciones y procedimientos que se usarán en MAIN.
\item MAIN: Empieza por START y acaba en END, es el bloque que contiene las instrucciones que van a ejecutarse y cierra el código.
\end{itemize}
De este modo, un archivo, punto de entrada a la gramática, tendría esta forma:
\begin{align*}
&ARCHIVO \rightarrow DECLARE\quad DEFINE\quad MAIN\\
\end{align*}
\subsection{Cierre de expresiones y bloques}
Las expresiones se acaban con '';'' para determinar donde acaba la expresión. Alternativamente, el final de un bloque lleva implícito el '';''.\\

Los bloques están delimitados por llaves ( elementos ''\{'' y ''\}''\ ) y se permite que estén anidados.\\

Para evitar errores, este lenguaje no permitirá que los '';'' y las llaves ''\{'' y ''\}'' estén en una linea distinta a la que deberían estar. Luego, si se detecta un salto de linea cuando debería haber uno de estos símbolos, saltará un error y no compilará, aunque éste esté en la linea siguiente. Esto conlleva que no se permiten las instrucciones multilineas para mejorar la claridad del código, sino que se deberá poner todo en la misma línea.
\subsection{Comentarios}
Los comentarios sólo pueden ser de una linea: empiezan con ''//'' y llegan hasta el final de línea.
\subsection{Declaración de variables y constantes}
Las variables se declaran poniendo el tipo de la variable seguido del identificador de la misma.\\

Los identificadores de variables y funciones empiezan siempre por letra, ya sea mayúscula o minúscula.\\

Los arrays se indican en la declaración de la variable poniendo después del tipo ''[longitud]'', donde ''longitud'' debe de ser un número. Al usarlos en expresiones, se debe incluir después del identificador de la variable, entre corchetes, el índice que se está usando del array. Para referirse al array entero (por ejemplo en un paso de parametros a una función) se usa el identificador de la variable sin los corchetes.\\

Un array de varias dimensiones lo que tiene son varios corchetes, uno por dimensión, después del identificador de variable (o del tipo cuando se declara la variable).\\
Finalmente, se pueden declarar constantes añadiendo a la declaración de tipos la palabra ''const'' justo antes del tipo.
\subsection{Expresiones}
Las expresiones podrán contener:
\begin{itemize}
\item Identificadores: en su primera aparición se les debe asignar un tipo poniendo el nombre del mismo antes del identificador de la variable.
\item Operadores:
	\begin{itemize}
	\item Suma: +\ (infijo, entre tipos numerales).
	\item Resta: -\ (infijo, entre tipos numerales).
	\item Multiplicación: *\ (infijo, entre tipos numerales).
	\item División: /\ (infijo, entre tipos numerales).
	\item Módulo: \% (infijo, entre tipos numerales).
	\item Potencia: \^\ (infijo, entre tipos numerales).
	\item Concatenación: + (infijo, entre cadenas de carácteres).
	\end{itemize}
\item Comparadores:
	\begin{itemize}
	\item Comparación de igualdad: ==\ (infijo).
	\item Comparación de desigualdad: !=\ (infijo).
	\item Comparación de menor o igual: <=\ (infijo, entre tipos numerales).
	\item Comparación de mayor o igual: >=\ (infijo, entre tipos numerales).
	\item Comparación de menor: <\ (infijo, entre tipos numerales).
	\item Comparación de mayor: >\ (infijo, entre tipos numerales).
	\item AND lógico: \&\&\ (infijo, entre tipos booleanos).
	\item OR lógico: ||\ (infijo, entre tipos booleanos).
	\end{itemize}
\item Asignaciones: usando =\ (infijo).
\item Asignaciones compuestas: usando op=\ (infijo) donde op es una operación aplicable al tipo de la asignación.
\end{itemize}

Todas estas expresiones funcionan en resumen como lo hacen en cualquier otro lenguaje como C o Java.
\newpage

\subsection{Instrucciones de control}
Las instrucciones de control habilitadas son:
\begin{itemize}
\item Condicional con una rama. Su estructura es:\\
		if (condición) \{ \\
			// Rama 1.\\
		\}
\item Condicional con dos ramas. Su estructura es:\\
		if (condición) \{ \\
			// Rama 1.\\
		\} else \{ \\
			// Rama 2.\\
		\}
\item Bucle While. Su estructura es:\\
		while (condición) \{ \\
			// Contenido del bucle.\\
		\}
\item Bucle Do-While. Su estructura es:\\
		 do \{ \\
			// Contenido del bucle.\\
		\} while (condición);
\item Bucle For. Su estructura es:\\
		for (integer i = 0; condición(i); i = i + 1) \{ \\
			// Contenido del bucle.\\
		\}
\item De salto. Su estructura es: break;\\
Simplemente salta fuera del bloque de control en el que se encuentra.
\item Condicional Case. Su estructura es:\\
		switch elemento \{ \\
		case comparación1: // Rama 1.\\
			break;\\
			\vdots\\
		default: // Rama por defecto.\\
			break;\\
		\}
\item De retorno: su estructura es: return;\\
Acaba una función y señala el valor que esta devuelve.
\end{itemize}
\subsection{Funciones y procedimientos}
Las funciones y procedimientos se tratan igual, la diferencia es que un procedimiento es una función que devuelve un valor de tipo void. La sintaxis es:\\

tipoDeRetorno nombreFunción(tipoVar1 var1, tipoVar2 var2, ... tipoVarN varN) \{ \\
// Cuerpo de la función.\\
\} \\

Para volver de una función o procedimiento se usa la instrucción return. Las que devuelven algún valor necesitan que en return se especifique la variable que contiene el valor (o el valor directamente) que se devuelve.
\subsection{Tipos}
Los tipos predefinidos son:
\begin{itemize}
\item Tipo número entero: int.
\item Tipo número en coma flotante: float.
\item Tipo expresión lógica: bool.
\item Tipo carácter: char.
\item Tipo cadena de carácteres: string.
\item Tipo vacío: void.
\end{itemize}

Para que un usuario defina sus propios tipos sólo se permite la creación de enumerados, con la siguiente estructura:\\

enum nombreDelEnumerado \{ \\
elemento1,\\
elemento2,\\
\vdots\\
elementoN\\
\} \\

Estos enumerados sirven principalmente para identificar una serie de valores numéricos con unos identificadores más explícitos y entendibles (como todos los enumerados).
\newpage

\section{Ejemplos de programas}
A continuación se muestran una serie de ejemplos de programas típicos realizados con este lenguaje.\footnote{A todos los programas les falta la identación porque no he sido capaz de que me funcione en \LaTeX}
\subsection{Hola Mundo}
Al no tener un sistema de entrada/salida, el hola mundo consistirá en escribir dicha frase en una cadena.\\

int main () \{ \\
string cadena;\\
cadena = "Hola mundo";\\
return 0;\\
\}

\subsection{Suma}
int main () \{ \\
int a = 10;\\
int b = 5;\\
int suma = a + b;\\
return 0;\\
\}

\subsection{Potencia (con función)}
int main () \{ \\
int a = 10;\\
int b = 2;\\
int pow = pow(a, b);\\
return 0;\\
\}\\

int pow(int a, int b) \{ \\
return a \^\ b;\\
\}
\newpage

\subsection{Factorial}
int main () \{ \\
int a = 0;\\
int f = 10;\\
a = f - 1;\\
while (a > 0) \{ \\
f = factorial(a, f);\\
a = a - 1;\\
\} \\
return 0;\\
\}\\

int factorial(int a, int f) \{ \\
return f * a;\\
\}

\subsection{Programa estandar}
int main () \{ \\
// Declaración de variables\\
while (true) \{ \\
if (func1) \{ \\
funcRespuesta1;\\
\} \\
if (func2) \{ \\
funcRespuesta2;\\
\} \\
\vdots\\
if (funcN) \{ \\
funcRespuestaN;\\
\} \\
\} \\
return 0;\\
\}

\end{document}